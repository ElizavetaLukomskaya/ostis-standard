\begin{SCn}
	
	\scnsectionheader{\currentname}
	
	\scnstartsubstruct
	
	\scnheader{Предметная область и онтология семантических технологий}
	
	\newpage
	\scnsegmentheader{Введение}

	\scneqfile{В марте 1989 года Тим Бернерс-Ли предложил концепцию распределенной информационной системы с целью "объединения знаний человечества", которую он назвал \textit{"Всемирной паутиной"} (\textit{World Wide Web} - \textit{WWW}). Для её создания он объединил две существующие технологии - \textit{технологию применения IP-протоколов для передачи данных} и \textit{технологию гипертекста} (\textit{Hypertext Technology}).
	Основными элементами технологии \textit{WWW} являются :
	\begin{scnitemize}
		\item \textit{язык гипертекстовой разметки документов} (\textit{Hyper Text Markup Language} - \textit{HTML});
		\item \textit{протокол обмена гипертекстовой информацией} (\textit{Hyper Text Transfer Protocol} - \textit{HTTP});
		\item \textit{универсальный способ адресации ресурсов в сети} (\textit{Universal Resource Identifier} - \textit{URI}, и \textit{Universal Resource Locator} - \textit{URL});
		\item \textit{система доменных имен} (\textit{Domain Name System} - \textit{DNS});
		\item \textit{универсальный интерфейс шлюзов} (\textit{Common Gateway Interface} - \textit{CGI}), добавленный позже сотрудниками \textit{Национального Центра Суперкомпьютерных Приложений} (\textit{National Center for Supercomputing Applications} - \textit{NCSA});
		\item \textit{расширяемый язык разметки} (\textit{eXtensible Markup Language} - \textit{XML}), рекомендованный \textit{Консорциумом Всемирной паутины}.
	\end{scnitemize}
	
	Основная цель \textbf{\textit{Semantic Web}} создать надстройку над существующей \textit{Всемирной паутиной}, позволяющую сделать размещаемую в \textit{Интернете} информацию пригодной для машинной обработки, а именно для автоматического анализа, синтеза выводов и преобразования как самих данных, так и сделанных на их основе заключений в различные представления, полезные на практике. 
	
	В частности, такой подход позволит повысить релевантность предоставляемой пользователю информации, обеспечить предоставление только необходимой части информации, облегчить поиск информации и повысить эффективность ее использования.
	
	Важно отметить, что концепции \textit{Web 2.0} и \textit{Web 3.0}, описывающие эволюцию интернет-технологий, не связаны непосредственно с \textit{\textbf{Semantic Web}}, более того, автор концепции \textit{Web 2.0} Тим О’Рейли неоднократно выступал с критикой отождествления \textit{\textbf{Semantic Web}} и \textit{Web 3.0}.}
\bigskip

\scnendsegmentcomment{Введение}
	
	\newpage
	\scseparatedfragment{Представление знаний}
\scnheader{Представление знаний}

\scnsegmentheader{Дескриптивная логика}

\scnstartsubstruct

\scnheader{дескриптивная логика}

\scnidtf{дескрипционная логика}
\scnidtf{descriptive logic}
\scnidtf{DL}
\scnidtfdef{семейство\textit{ формальных языков представления знаний}.} 

\scnsubdividing{DL общего вида; темпоральные DL; пространственные DL; пространственно-темпоральные DL; нечеткие DL}

\scnsuperset{концепт}
\scnaddlevel{1}
\scnidtf{одноместный предикат}
\scnidtf{множество}
\scnidtf{класс}
\scnaddlevel{-1}

\scnsuperset{роль}
\scnaddlevel{1}
\scnidtf{двухместный предикат}
\scnidtf{бинарное отношение}
\scnaddlevel{-1}


\scnnote{Большинство \textit{DL} более выразительны, чем \textit{пропозициональные логики}, но менее выразительны, чем\textit{ логика предикатов первого порядка}. В отличие от последних, \textit{DL} обычно разрешимы, для них определены эффективные \textit{процедуры вывода}. В целом в \textit{DL} обычно соблюдается баланс между выразительной мощностью и сложностью организации вывода. Современное название семейство \textit{DL} получило в 1980-е годы, тогда они изучались как расширения \textit{теорий фреймовых структур} и \textit{семантических сетей} механизмами \textit{формальной логики}. В 2000-е годы \textit{дескрипционные логики} получили применение в рамках концепции \textit{Semantic Web паутины}, где их предлагалось использовать при построении \textit{онтологий}. На основе \textit{DL} построены подъязыки \textit{OWL} такие как \textit{OWL-DL} и \textit{OWL-Lite}.}

\scnheader{ALC}
\scnidtf{attributive language with complement}
\scnidtfdef{одна из базовых систем, на основе которой строятся многие другие \textit{дескрипционные логики}}
\scnidtfdef{семейство \textit{логик}, где каждая \textit{логика} этого семейства задается выбором конкретных \textit{множеств} атомарных \textit{концептов} и \textit{ролей}}

\scnsuperset{концепты логики}
\scnaddlevel{1}

\scnrelfromset{индуктивное определение}{
	\scnfileitem{всякий атомарный \textit{концепт} является \textit{концептом}};
	\scnfileitem{выражения \(\top\) и \(\bot\) являются \textit{концептами}};
	\scnfileitem{если C есть \textit{концепт}, то его дополнение ¬C является \textit{концептом}};
	\scnfileitem{если C и D есть \textit{концепты}, то их \textit{пересечение} C \(\wedge\) D и \textit{объединение} C \(\lor\) D являются \textit{концептами}};
	\scnfileitem{если C есть \textit{концепт}, а R есть \textit{роль}, то выражения \(\forall\) R. C и \(\exists\)R. C R.C являются \textit{концептами}}}
	
	\scnaddlevel{1}
	\scnexplanation{Пояснения к индуктивным определениям концептов логики ALC:
		
		\begin{scnitemize}
		\item c точки зрения \textit{интерпретации} (\textit{семантики}) \(\top\) трактуется как множество всевозможных \textit{сущностей} (весь \textit{домен}), а \(\bot\) как пустое \textit{множество};
				
		\item каждый \textit{концепт} A трактуется как \textit{подмножество} \textit{домена}, каждая \textit{роль} - как \textit{бинарное отношение} на \textit{домене}, то есть R \(\subseteq\) \(\top\) \(\times\) \(\top\);
		
		\item \textit{дополнение}, \textit{пересечение} и \textit{объединение} \textit{концептов} трактуются как аналогичные \textit{теоретико-множественные операции};
		
		\item выражение \(\forall\)R.C интерпретируется как \textit{множество} всех \textit{индивидов}, которые связаны \textit{отношением} R только с \textit{индивидами} \textit{концепта} C;
		
		\item выражение \(\exists\)R.C интерпретируется как \textit{множество} тех \textit{индивидов}, которые связаны \textit{отношением} R с каким-либо \textit{индивидом} \textit{концепта} C.
	    \end{scnitemize}
    }

	\scnnote{Для определения \textit{концептов логики ALC} допускаем, что заданы непустые конечные \textit{множества} атомарных \textit{концептов} и атомарных \textit{ролей}. В свою очередь, семантика \textit{дескрипционных логик} задается путём \textit{интерпретации} её атомарных \textit{концептов} как множеств объектов («\textit{индивидов}»), выбираемых из некоторого фиксированного множества («\textit{домена}»), а атомарных \textit{ролей} — как множеств пар индивидов, то есть \textit{бинарных отношений} на \textit{домене}. Из этого следует, что \textit{классические DL} не предполагают наличия произвольных отношений между \textit{концептами} (\textit{классами}). Формально интерпретация какой-либо \textit{DL} семейства \textit{ALC} задается \textit{доменом} (множеством всевозможных \textit{сущностей}) и \textit{интерпретирующей функцией}, задающей \textit{концепты} и \textit{роли} на основе данного домена.}
	\scnaddlevel{-1}

\scnaddlevel{-1}


\scnheader{База знаний в контексте DL}
\scnidtfdef{\textit{объединение} двух \textit{множеств}:
	\begin{scnitemize}
		\item набор \textit{терминологических аксиом} или \textit{TBox}
		\item набор \textit{утверждений} об \textit{индивидах} или \textit{ABox}
	\end{scnitemize}
}

\scnheader{Терминологические аксиомы}
\scnsubdividing{аксиома вложенности; аксиома эквивалентности}
		\scnaddlevel{1}
		\scnexplanation{
			\begin{scnitemize}
			\item \textit{аксиома вложенности} -  C \(\subseteq\) D (как для \textit{концептов}, так и для \textit{ролей})
			
			\item \textit{аксиома эквивалентности} - C \(\equiv\) D (как для \textit{концептов}, так и для \textit{ролей})
			\end{scnitemize}
		}
		\scnaddlevel{-1}
	
\scnnote{	
	Пример:
	\textit{Например, следующая совокупность является терминологией (или TBox) для какой-либо DL семейства ACL:}\newline
		
		\textit{Woman \(\equiv\) Person \(\cap\) Female \newline
		Mother \(\equiv\) Woman \(\cap\) \(\exists\) hasChild.\(\top\) \newline
		\(\forall\) hasChild.Person \(\subseteq\) Person \newline
		Doctor \(\subseteq\) Person\newline}


Данный \textit{фрагмент} может интерпретироваться следующим образом:
\begin{scnitemize}
\item быть женщиной означает в точности быть человеком и быть женского пола; 
\item быть матерью означает в точности быть женщиной и иметь ребёнка; 
\item у всякого человека всякий ребёнок есть тоже человек; 
\item всякий доктор является человеком.
\end{scnitemize}

Если \textit{терминологическая аксиома} выполняется в рамках какой-либо \textit{интерпретации}, то говорят, что данная \textit{интерпретация} является \textit{моделью} для данной \textit{аксиомы}. \textit{Моделью} заданного \textit{TBox} называется \textit{интерпретация}, которая является \textit{моделью} для всех \textit{аксиом} данного \textit{TBox}.}



\scnheader{Утверждения об индивидах}
\scnsubdividing{утверждение о принадлежности индивида концепту; утверждение о связи двух индивидов ролью}

		\scnaddlevel{1}
		\scnexplanation{
			\begin{scnitemize}
				\item утверждение о принадлежности индивида a концепту C — записывается как C(a) или a:C;
				
				\item утверждение о связи двух индивидов a и b ролью R — записывается как R(a,b) или (a,b):R или aRb.
			\end{scnitemize}
		}
		\scnaddlevel{-1}

\scnnote{	
Пример:
\textit{Например, следующая совокупность является набором утверждений об индивидах (ABox) для какой-либо DL семейства ACL:}\newline
\textit{
Mary : Woman \(\cap\) ¬Doctor\newline
Mary : \(\exists\) hasChild.Female \newline
Mary: hasChild Peter\newline
Peter : Doctor \(\cap\) \(\forall\) hasChild . \(\bot\)}\newline

Здесь \textit{Mary} и \textit{Peter} есть имена \textit{индивидов}. Эти утверждения означают, что \textit{Mary} является женщиной, но не доктором, у неё есть ребёнок женского пола, \textit{Peter} также является ребёнком \textit{Mary}, причем \textit{Peter} является доктором и не имеет детей.
\textit{Моделью} заданного \textit{ABox} является \textit{интерпретация}, в которой выполняются все \textit{утверждения} данного \textit{ABox}.

В рамках \textit{теории DL} существует большое количество \textit{расширений ACL}, основными из которых являются следующие (\textit{R-последователем} называется \textit{второй компонент пары}, принадлежащей \textit{отношению} R, для заданного \textit{первого компонента}):
\begin{scnitemize}
\item \textit{F} - \textit{Функциональность ролей}: \textit{концепты} вида ( \(\leq\) 1 R ), означающие: существует не более одного \textit{R-последователя};
\item \textit{N} - \textit{Ограничения кардинальности ролей}: \textit{концепты} вида ( \(\leq\) n R ) , означающие: существует не более n \textit{R-последователей};
\item \textit{Q} - \textit{Качественные ограничения кардинальности ролей}: концепты вида ( \(\leq\) n R.C ), означающие: существует не более n \textit{R-последователей} в \textit{C};    
\item \textit{I} - \textit{Обратные роли}: если R есть \textit{роль}, то R − тоже является \textit{ролью}, означающей \textit{обращение бинарного отношения};
\item \textit{O} - \textit{Номиналы}: если a есть \textit{имя индивида}, то \{ a \} есть \textit{концепт}, означающий \textit{одноэлементное множество};
\item \textit{H} - \textit{Иерархия ролей}: в \textit{TBox} допускаются \textit{аксиомы вложенности ролей}    R \(\subseteq\) S
\item \textit{S} - \textit{Транзитивные роли}: в \textit{TBox} допускаются \textit{аксиомы транзитивности} вида Tr(R)
\item \textit{R} - \textit{Составные аксиомы вложенности ролей} в \textit{TBox} ( R \(\circ\) S \(\subseteq\) R, R \(\circ\) S \(\subseteq\) S) с условием \textit{ацикличности}, где R \(\circ\) S есть \textit{композиция ролей};
\item \textit{(D)} - \textit{Расширение языка конкретными доменами} (\textit{типами данных})
\end{scnitemize}

Исследуются и используются различные \textit{совокупности} этих \textit{свойств}. Например, \textit{язык OWL-Lite} соответствует \textit{дескрипционной логике SHIF(D)}, а \textit{язык OWL-DL} соответствует \textit{дескрипционной логике SHOIN(D)}.}

\scnheader{Логический вывод в DL}

\scnrelfrom{реализация}{ризонер}
\scnaddlevel{1}
\scnexplanation{Каждый из \textit{ризонеров} поддерживает определенный \textit{набор расширений ACL}, что оговаривается в его \textit{спецификации}.}
\scnaddlevel{-1}

\scnsuperset{логический анализ}
\scnaddlevel{1}
\scnidtfdef{\textit{концепт} C данной логики выполняется в \textit{интерпретации} I, если в этой \textit{интерпретации} у него есть хотя бы 1 \textit{индивид}}
\scnidtfdef{\textit{концепт} C называется \textit{выполнимым}, если существует \textit{интерпретация}, в которой он выполняется}
\scnidtfdef{\textit{концепт} C вложен в \textit{концепт} D (или содержится в нём; англ. is subsumed by), если в любой \textit{интерпретации} I выполняется C \(\subseteq\) D}
\scnaddlevel{-1}

\scnnote{С учетом определений ключевые проблемы, решаемые при помощи \textit{логического вывода в DL}, обычно следующие:
	\begin{scnitemize}
	\item выполнимость \textit{концепта}: является ли заданный \textit{концепт} \textit{выполнимым} относительно заданного \textit{TBox}?
	\item вложенность \textit{концептов}: верно ли, что один заданный \textit{концепт} вложен в другой относительно заданного \textit{TBox}?
	\item совместимость \textit{TBox}: имеет ли заданный \textit{TBox} хотя бы одну \textit{модель}?
	\item совместимость \textit{базы знаний}: имеет ли заданная \textit{пара} (\textit{TBox}, \textit{ABox}) хотя бы одну \textit{модель}?
	\end{scnitemize}

Кроме того, важное практическое значение имеют такие проблемы, как:
\begin{scnitemize}
\item \textit{классификация терминологии}: для данной \textit{терминологии} (то есть \textit{TBox}) построить \textit{таксономию} или \textit{иерархию концептов}, то есть упорядочить все \textit{атомарные концепты} по отношению вложения (отн. данного \textit{TBox}) и выдать соответствующее \textit{частично упорядоченное множество};
\item извлечение \textit{экземпляров концепта}: найти все \textit{экземпляры} заданного \textit{концепта} относительно заданной \textit{базы знаний};
\item наиболее узкий \textit{концепт} для \textit{индивида}: найти наименьший (по вложению) \textit{концепт}, \textit{экземпляром} которого является заданный \textit{индивид} относительно заданной \textit{базы знаний};
\item ответ на \textit{запрос} к \textit{базе знаний}: выдать все наборы \textit{индивидов}, которые удовлетворяют заданному \textit{запросу} относительно заданной \textit{базы знаний}.
\end{scnitemize}}

\scnrelfromlist{выводы к сегменту}{
	\scnfileitem {\textit{Дескриптивные логики} изначально создавались как способ \textit{представления знаний}, обладающий балансом между выразительной мощностью и вычислительной сложностью;};
	\scnfileitem {Сами по себе классические \textit{DL} достаточно ограничены и в основном ориентированы на \textit{представление таксономий} (\textit{иерархий}) \textit{классов}, а также описании \textit{свойств} между \textit{экземплярами} этих \textit{классов}. Фактически классические \textit{DL} не содержат \textit{средств описания свойств} самих \textit{классов} и \textit{отношений} между ними, \textit{средств структуризации баз знаний}, \textit{средств записи }нечетких, нестационарных и других \textit{знаний} с \textit{НЕ-факторами}.}}





\bigskip

\scnendstruct \scnendsegmentcomment{Дескриптивная логика}
\bigskip
\scnendcurrentsectioncomment
	
	\newpage
	\scseparatedfragment{Обработка информации}
\scnheader{Обработка информации}

\scnstartsubstruct
\scnnote{Ключевыми \textit{задачами} \textit{обработки информации}, представленной при помощи \textit{средств} \textit{\textbf{Semantic Web}}, являются:
	\begin{scnitemize}
		\item Выполнение \textit{поисковых запросов} (\textit{Query});
		\item \textit{Логический вывод} на \textit{онтологиях} (порождения новой \textit{информации} на основе имеющейся).
	\end{scnitemize}}

\scnsegmentheader{Языки запросов}

\scnstartsubstruct

\scnexplanation{Для выполнения \textit{запросов} к \textit{хранилищу}, содержащему \textit{информацию}, записанную в \textit{RDF}, используются \textit{языки запросов}(\textit{Query languages})}

\scnheader{SPARQL}
\scnidtf{SPARQL Protocol and RDF Query Language}
\scnidtfdef{\textit{язык запросов}, признанный в качестве \textit{стандарта} W3C в 2008 году}

\scnrelfrom{поддерживаемые операции}{операции SQL}
	\scnaddlevel{1}
	\scnsubdividing{объединение (JOIN);сортировка (SORT)}
	\scnaddlevel{-1}

\scnnote{\textit{Запрос} \textit{SPARQL} по сути представляет собой \textit{шаблон} некоторого\textit{ RDF-графа}, в котором неизвестные \textit{сущности} задаются \textit{переменными}.
\textit{SPARQL} позволяет получить \textit{результат запроса} в нескольких формах:
\begin{scnitemize}
	\item \textit{Оператор} \textit{SELECT} позволяет указать, значения каких \textit{переменных} мы хотели бы получить, \textit{результат} возвращается в виде \textit{таблицы};
	\item \textit{Оператор CONSTRUCT} позволяет получить \textit{ответ} в виде \textit{RDF-графа};
	\item \textit{Оператор ASK} позволяет получить \textit{ответ} вида да/нет (существует ли вообще такая \textit{конструкция});
	\item \textit{Оператор CONSTRUCT} позволяет получить \textit{ответ} в виде \textit{RDF-графа}, \textit{содержание} которого может дополнительно уточняться при \textit{запросе};
\end{scnitemize}


Например:

\textit{PREFIX foaf: <http://xmlns.com/foaf/0.1/>\newline
SELECT ?name\newline
?email\newline
WHERE\newline
{\newline
	?person  a   foaf:Person .\newline
	?person  foaf:name  ?name .\newline
	?person  foaf:mbox  ?email .\newline
}}\newline

Данный запрос позволяет получить \textit{имена} и \textit{e-mail} всех \textit{персон}, известных \textit{системе}.

\textit{PREFIX ex: <http://example.com/exampleOntology#>\newline
SELECT ?capital\newline
?country\newline
WHERE\newline
\{\newline
	?x  ex:cityname   	?capital   ;\newline
	ex:isCapitalOf	?y     	.\newline
	?y  ex:countryname	?country   ;\newline
	ex:isInContinent  ex:Africa  .\newline
\}}\newline

Данный запрос позволяет получить названия всех столичных городов в Африке.\\

Ссылка:
\scncite{https://www.w3.org/TR/sparql11-query/}}

\scnheader{GraphQL}
\scnidtfdef{открытый \textit{язык} для различного рода \textit{графовых хранилищ} и \textit{СУБД}}
\scnrelfrom{синтаксическая схожесть}{JSON}
\scnrelfromlist{оперирование понятиями}{поле; значение поля}
\scnrelto{разработчик}{\textit{\textbf{Facebook}}}
\scnnote{\textit{Язык} \textit{GraphQL} был предложен компанией Facebook в 2012 году. Существует большое число реализаций, поддерживающих \textit{GraphQL}.
	
Ссылка:
\scncite{https://graphql.org/} }

\scnheader{Gremlin}
\scnidtfdef{популярный \textit{язык обхода графа} (\textit{graph traversal language})}
\scnrelto{разработчик}{\textit{\textbf{Apache}}}
\scnnote{Gremlin позволяет описывать \textit{запросы} к \textit{хранилищу графов} как в виде \textit{шаблонов} с \textit{переменными}, так и более простым способом, указывая необходимые \textit{свойства} искомых \textit{компонентов графа}. Поддержка \textit{Gremlin} реализована в большинстве современных \textit{хранилищ}.
	
Пример:
\textit{g.V().hasLabel('movie').values('year').min()}
 
Данный \textit{запрос} позволяет найти все \textit{вершины графа}, имеющие \textit{метку} “фильм” и выбрать минимальный по году выпуска.}

\scnheader{Cypher}
\scniselement{открытый язык}
\scnidtfdef{\textit{язык запросов} для \textit{графовой СУБД Neo4j}}
\scnrelfromlist{результат запроса}{граф; таблица}
\scnaddlevel{2}
\scnexplanation{В случае, когда граф построить нельзя, например запрашиваются только значения каких-либо атрибута для заданного множества сущностей}
\scnaddlevel{-2}
\scnnote{Cypher позволяет в достаточно удобной форме описывать шаблоны, на основе которых осуществляется в поиск в графе. Как и в модели СУБД Neo4j в Cypher различаются отношения между сущностями (аналог ObjectProperty в OWL) и свойства сущностей (атрибуты, аналог DataProperty в OWL).

Пример:\newline
\textit{MATCH (nicole:Actor {name: 'Nicole Kidman'})-[:ACTED\_IN]->(movie:Movie)\newline
WHERE movie.year < 1990\newline
RETURN movie\newline}
Данный \textit{запрос} позволяет найти все фильмы старше 1990 года, в которых сыграла Николь Кидман.}

\bigskip

\scnendstruct \scnendsegmentcomment{Языки запросов}

\scnsegmentheader{Ризонеры}

\scnstartsubstruct

\scnheader{Ризонеры}
\scnidtf{reasoners}
\scnrelfrom{область применения}{логический вывод}
\scnaddlevel{2}
\scnexplanation{\textit{Ризонеры}, как правило, реализуют \textit{прямой логический вывод} на основе \textit{отношений}, описанных в \textit{онтологии}. В простейшем случае это \textit{вывод} на основе \textit{свойства транзитивности отношений}, например \textit{SubClassOf}, в более сложных могут учитываться \textit{правила}, записанные, например в \textit{SWRL}.}
\scnaddlevel{-2}

\scnnote{Часто для \textit{ризонера} оговаривается \textit{язык}, \textit{семантика} которого полностью поддерживается соответствующим \textit{ризонером}, например \textit{OWL Lite} или \textit{OWL DL}.\newline
	
Полный список признанных сообществом \textit{ризонеров} с кратким описанием их возможностей приведен в статье:
\scncite{https://www.w3.org/2001/sw/wiki/OWL/Implementations/}}

\bigskip

\scnendstruct \scnendsegmentcomment{Ризонеры}

\bigskip

\scnrelfromlist{выводы к разделу}{
	\scnfileitem {Интересно отметить, что все рассмотренные \textit{языки запросов} (включая \textit{SPARQL}, который является \textit{стандартом W3C}), кроме \textit{GraphQL}, позволяют осуществлять только \textit{поиск} (собственно \textit{запросы}), и не дают возможности что-либо создавать, редактировать или удалять в \textit{графе}, в отличие, например, от \textit{языка} \textit{SQL}, обладающем соответсвующими \textit{классами операторов}. Это означает, что фактически эта часть никак не стандартизируется и отдается на откуп разработчикам конкретного \textit{хранилища}, которые формируют соответствующий \textit{программный интерфейс}.};
	\scnfileitem {\textit{Возможности} и \textit{принципы} \textit{логического вывода} на \textit{онтологиях} зависят на \textit{синтаксиса} и \textit{семантики} \textit{языков}, используемых для \textit{представления знаний}. В остальном \textit{обработка информации}, записанной \textit{средствами} \textit{\textbf{Semantic Web}} никак не стандартизируется - реализуются \textit{программные интерфейсы} для \textit{доступа} к \textit{хранилищам} для различных \textit{языков программирования}, после чего разработчик конкретной \textit{системы} сам определяет, как использовать полученную \textit{информацию}.}}

\bigskip
\scnendstruct 
\scnendcurrentsectioncomment
	
	\newpage
	\scseparatedfragment{Направления развития Semantic Web}
\scnheader{Направления развития Semantic Web}

\scneqfile{Несмотря на довольно очевидную выгоду от ее \textit{внедрения}, \textit{общая концепция Semantic Web} в настоящее время не реализована даже на базовом уровне. Уже в 2006 году Тим Бернерс-Ли опубликовал статью\textit{ “Semantic Web Revisited”}, в которой признал, что по ряду причин \textit{концепция} в большой степени не реализована.\newline
	Причин такой ситуации несколько:
	\begin{scnitemize}
		\item человеческий фактор и накладные расходы - описание \textit{метаданных} для какого-либо \textit{ресурса} и вообще его \textit{формализация} требует дополнительных затрат, иногда довольно существенных, но в то же время не дает очевидной прямой выгоды в ближайшей перспективе. Кроме того, многие коммерческие компании не заинтересованы в \textit{стандартизации} подобных \textit{решений} в принципе, поскольку это снизит объем выполняемой работы и востребованность таких компаний на рынке;
		\item сами по себе \textit{средства представления} и \textit{обработки знаний \textbf{Semantic Web}} достаточно сложны в использовании для неподготовленного человека;
		\item остается открытым вопрос с выбором \textit{универсальной системы понятий верхнего уровня}, которые бы гарантированно позволили описать любую \textit{информацию}; кроме того, возникает вопрос о том, возможно ли вообще выбрать такую \textit{систему понятий};
		\item добавление \textit{метаинформации} автоматически приводит к частичному дублированию уже имеющейся \textit{информации} и соответствующим расходам.
	\end{scnitemize}
}
\scnendcurrentsectioncomment

\bigskip

\scnrelfromlist{выводы к разделу}{
\scnfileitem {\textit{Стандарты} \textit{\textbf{Semantic Web}} практически не меняются на протяжении более 15 лет. Это связано с тем, что \begin{scnitemize}\item сами \textit{стандарты} все еще описаны в традиционной  \textit{гипертекстовой форме}, которая достаточно трудоемка в сопровождении;\item существует огромное количество \textit{средств}, которые жестко завязаны на определенную \textit{версию} того или иного \textit{стандарта} и сопровождаются сторонними коллективами разработчиков. Соответственно, каждое изменение \textit{стандарта} должно отразиться и в указанных \textit{средствах}, что зачастую очень сложно;\item значительная часть \textit{средств}, построенных на основе \textit{стандартов}, напрямую зависит не только от \textit{семантики языков}, но и от конкретной \textit{формы записи} того или иного \textit{языка}, что еще усугубляет проблему из предыдущего пункта.\end{scnitemize}};
\scnfileitem {Фактически все \textit{семейство стандартов} \textit{\textbf{Semantic Web}} изначально создавалось для реализации идей \textit{семантической паутины}, т.е. описания \textit{метаинформации} для различных \textit{web-ресурсов}, приведение их к согласованному виду (как с синтаксическом, так и в семантическом плане) и т.д. Указанные \textit{стандарты} изначально не претендовали на всеобщую \textit{стандартизацию принципов представления} и \textit{обработки} любого рода \textit{информации} в \textit{интеллектуальных системах} произвольного назначения. Однако, де-факто в настоящее время они часто используются именно в такой роли, поскольку не существует более мощной и удобной альтернативы, общепризнанной в мировом сообществе.}}
	
	\bigskip
	\scnendstruct \scnendfragmentcomment
	
\end{SCn}

