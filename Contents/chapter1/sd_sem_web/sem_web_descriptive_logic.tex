\scseparatedfragment{Представление знаний}
\scnheader{Представление знаний}

\scnsegmentheader{Дескриптивная логика}

\scnstartsubstruct

\scnheader{дескриптивная логика}

\scnidtf{дескрипционная логика}
\scnidtf{descriptive logic}
\scnidtf{DL}
\scnidtfdef{семейство\textit{ формальных языков представления знаний}.} 

\scnsubdividing{DL общего вида; темпоральные DL; пространственные DL; пространственно-темпоральные DL; нечеткие DL}

\scnsuperset{концепт}
\scnaddlevel{1}
\scnidtf{одноместный предикат}
\scnidtf{множество}
\scnidtf{класс}
\scnaddlevel{-1}

\scnsuperset{роль}
\scnaddlevel{1}
\scnidtf{двухместный предикат}
\scnidtf{бинарное отношение}
\scnaddlevel{-1}


\scnnote{Большинство \textit{DL} более выразительны, чем \textit{пропозициональные логики}, но менее выразительны, чем\textit{ логика предикатов первого порядка}. В отличие от последних, \textit{DL} обычно разрешимы, для них определены эффективные \textit{процедуры вывода}. В целом в \textit{DL} обычно соблюдается баланс между выразительной мощностью и сложностью организации вывода. Современное название семейство \textit{DL} получило в 1980-е годы, тогда они изучались как расширения \textit{теорий фреймовых структур} и \textit{семантических сетей} механизмами \textit{формальной логики}. В 2000-е годы \textit{дескрипционные логики} получили применение в рамках концепции \textit{Semantic Web паутины}, где их предлагалось использовать при построении \textit{онтологий}. На основе \textit{DL} построены подъязыки \textit{OWL} такие как \textit{OWL-DL} и \textit{OWL-Lite}.}

\scnheader{ALC}
\scnidtf{attributive language with complement}
\scnidtfdef{одна из базовых систем, на основе которой строятся многие другие \textit{дескрипционные логики}}
\scnidtfdef{семейство \textit{логик}, где каждая \textit{логика} этого семейства задается выбором конкретных \textit{множеств} атомарных \textit{концептов} и \textit{ролей}}

\scnsuperset{концепты логики}
\scnaddlevel{1}

\scnrelfromset{индуктивное определение}{
	\scnfileitem{всякий атомарный \textit{концепт} является \textit{концептом}};
	\scnfileitem{выражения \(\top\) и \(\bot\) являются \textit{концептами}};
	\scnfileitem{если C есть \textit{концепт}, то его дополнение ¬C является \textit{концептом}};
	\scnfileitem{если C и D есть \textit{концепты}, то их \textit{пересечение} C \(\wedge\) D и \textit{объединение} C \(\lor\) D являются \textit{концептами}};
	\scnfileitem{если C есть \textit{концепт}, а R есть \textit{роль}, то выражения \(\forall\) R. C и \(\exists\)R. C R.C являются \textit{концептами}}}
	
	\scnaddlevel{1}
	\scnexplanation{Пояснения к индуктивным определениям концептов логики ALC:
		
		\begin{scnitemize}
		\item c точки зрения \textit{интерпретации} (\textit{семантики}) \(\top\) трактуется как множество всевозможных \textit{сущностей} (весь \textit{домен}), а \(\bot\) как пустое \textit{множество};
				
		\item каждый \textit{концепт} A трактуется как \textit{подмножество} \textit{домена}, каждая \textit{роль} - как \textit{бинарное отношение} на \textit{домене}, то есть R \(\subseteq\) \(\top\) \(\times\) \(\top\);
		
		\item \textit{дополнение}, \textit{пересечение} и \textit{объединение} \textit{концептов} трактуются как аналогичные \textit{теоретико-множественные операции};
		
		\item выражение \(\forall\)R.C интерпретируется как \textit{множество} всех \textit{индивидов}, которые связаны \textit{отношением} R только с \textit{индивидами} \textit{концепта} C;
		
		\item выражение \(\exists\)R.C интерпретируется как \textit{множество} тех \textit{индивидов}, которые связаны \textit{отношением} R с каким-либо \textit{индивидом} \textit{концепта} C.
	    \end{scnitemize}
    }

	\scnnote{Для определения \textit{концептов логики ALC} допускаем, что заданы непустые конечные \textit{множества} атомарных \textit{концептов} и атомарных \textit{ролей}. В свою очередь, семантика \textit{дескрипционных логик} задается путём \textit{интерпретации} её атомарных \textit{концептов} как множеств объектов («\textit{индивидов}»), выбираемых из некоторого фиксированного множества («\textit{домена}»), а атомарных \textit{ролей} — как множеств пар индивидов, то есть \textit{бинарных отношений} на \textit{домене}. Из этого следует, что \textit{классические DL} не предполагают наличия произвольных отношений между \textit{концептами} (\textit{классами}). Формально интерпретация какой-либо \textit{DL} семейства \textit{ALC} задается \textit{доменом} (множеством всевозможных \textit{сущностей}) и \textit{интерпретирующей функцией}, задающей \textit{концепты} и \textit{роли} на основе данного домена.}
	\scnaddlevel{-1}

\scnaddlevel{-1}


\scnheader{База знаний в контексте DL}
\scnidtfdef{\textit{объединение} двух \textit{множеств}:
	\begin{scnitemize}
		\item набор \textit{терминологических аксиом} или \textit{TBox}
		\item набор \textit{утверждений} об \textit{индивидах} или \textit{ABox}
	\end{scnitemize}
}

\scnheader{Терминологические аксиомы}
\scnsubdividing{аксиома вложенности; аксиома эквивалентности}
		\scnaddlevel{1}
		\scnexplanation{
			\begin{scnitemize}
			\item \textit{аксиома вложенности} -  C \(\subseteq\) D (как для \textit{концептов}, так и для \textit{ролей})
			
			\item \textit{аксиома эквивалентности} - C \(\equiv\) D (как для \textit{концептов}, так и для \textit{ролей})
			\end{scnitemize}
		}
		\scnaddlevel{-1}
	
\scnnote{	
	Пример:
	\textit{Например, следующая совокупность является терминологией (или TBox) для какой-либо DL семейства ACL:}\newline
		
		\textit{Woman \(\equiv\) Person \(\cap\) Female \newline
		Mother \(\equiv\) Woman \(\cap\) \(\exists\) hasChild.\(\top\) \newline
		\(\forall\) hasChild.Person \(\subseteq\) Person \newline
		Doctor \(\subseteq\) Person\newline}


Данный \textit{фрагмент} может интерпретироваться следующим образом:
\begin{scnitemize}
\item быть женщиной означает в точности быть человеком и быть женского пола; 
\item быть матерью означает в точности быть женщиной и иметь ребёнка; 
\item у всякого человека всякий ребёнок есть тоже человек; 
\item всякий доктор является человеком.
\end{scnitemize}

Если \textit{терминологическая аксиома} выполняется в рамках какой-либо \textit{интерпретации}, то говорят, что данная \textit{интерпретация} является \textit{моделью} для данной \textit{аксиомы}. \textit{Моделью} заданного \textit{TBox} называется \textit{интерпретация}, которая является \textit{моделью} для всех \textit{аксиом} данного \textit{TBox}.}



\scnheader{Утверждения об индивидах}
\scnsubdividing{утверждение о принадлежности индивида концепту; утверждение о связи двух индивидов ролью}

		\scnaddlevel{1}
		\scnexplanation{
			\begin{scnitemize}
				\item утверждение о принадлежности индивида a концепту C — записывается как C(a) или a:C;
				
				\item утверждение о связи двух индивидов a и b ролью R — записывается как R(a,b) или (a,b):R или aRb.
			\end{scnitemize}
		}
		\scnaddlevel{-1}

\scnnote{	
Пример:
\textit{Например, следующая совокупность является набором утверждений об индивидах (ABox) для какой-либо DL семейства ACL:}\newline
\textit{
Mary : Woman \(\cap\) ¬Doctor\newline
Mary : \(\exists\) hasChild.Female \newline
Mary hasChild Peter\newline
Peter : Doctor \(\cap\) \(\forall\) hasChild . \(\bot\)}\newline

Здесь \textit{Mary} и \textit{Peter} есть имена \textit{индивидов}. Эти утверждения означают, что \textit{Mary} является женщиной, но не доктором, у неё есть ребёнок женского пола, \textit{Peter} также является ребёнком \textit{Mary}, причем \textit{Peter} является доктором и не имеет детей.
\textit{Моделью} заданного \textit{ABox} является \textit{интерпретация}, в которой выполняются все \textit{утверждения} данного \textit{ABox}.

В рамках \textit{теории DL} существует большое количество \textit{расширений ACL}, основными из которых являются следующие (\textit{R-последователем} называется \textit{второй компонент пары}, принадлежащей \textit{отношению} R, для заданного \textit{первого компонента}):
\begin{scnitemize}
\item \textit{F} - \textit{Функциональность ролей}: \textit{концепты} вида ( \(\leq\) 1 R ), означающие: существует не более одного \textit{R-последователя};
\item \textit{N} - \textit{Ограничения кардинальности ролей}: \textit{концепты} вида ( \(\leq\) n R ) , означающие: существует не более n \textit{R-последователей};
\item \textit{Q} - \textit{Качественные ограничения кардинальности ролей}: концепты вида ( \(\leq\) n R.C ), означающие: существует не более n \textit{R-последователей} в \textit{C};    
\item \textit{I} - \textit{Обратные роли}: если R есть \textit{роль}, то R − тоже является \textit{ролью}, означающей \textit{обращение бинарного отношения};
\item \textit{O} - \textit{Номиналы}: если a есть \textit{имя индивида}, то \{ a \} есть \textit{концепт}, означающий \textit{одноэлементное множество};
\item \textit{H} - \textit{Иерархия ролей}: в \textit{TBox} допускаются \textit{аксиомы вложенности ролей}    R \(\subseteq\) S
\item \textit{S} - \textit{Транзитивные роли}: в \textit{TBox} допускаются \textit{аксиомы транзитивности} вида Tr(R)
\item \textit{R} - \textit{Составные аксиомы вложенности ролей} в \textit{TBox} ( R \(\circ\) S \(\subseteq\) R, R \(\circ\) S \(\subseteq\) S) с условием \textit{ацикличности}, где R \(\circ\) S есть \textit{композиция ролей};
\item \textit{(D)} - \textit{Расширение языка конкретными доменами} (\textit{типами данных})
\end{scnitemize}

Исследуются и используются различные \textit{совокупности} этих \textit{свойств}. Например, \textit{язык OWL-Lite} соответствует \textit{дескрипционной логике SHIF(D)}, а \textit{язык OWL-DL} соответствует \textit{дескрипционной логике SHOIN(D)}.}

\scnheader{Логический вывод в DL}

\scnrelfrom{реализация}{ризонер}
\scnaddlevel{1}
\scnexplanation{Каждый из \textit{ризонеров} поддерживает определенный \textit{набор расширений ACL}, что оговаривается в его \textit{спецификации}.}
\scnaddlevel{-1}

\scnsuperset{логический анализ}
\scnaddlevel{1}
\scnidtfdef{\textit{концепт} C данной логики выполняется в \textit{интерпретации} I, если в этой \textit{интерпретации} у него есть хотя бы 1 \textit{индивид}}
\scnidtfdef{\textit{концепт} C называется \textit{выполнимым}, если существует \textit{интерпретация}, в которой он выполняется}
\scnidtfdef{\textit{концепт} C вложен в \textit{концепт} D (или содержится в нём; англ. is subsumed by), если в любой \textit{интерпретации} I выполняется C \(\subseteq\) D}
\scnaddlevel{-1}

\scnnote{С учетом определений ключевые проблемы, решаемые при помощи \textit{логического вывода в DL}, обычно следующие:
	\begin{scnitemize}
	\item выполнимость \textit{концепта}: является ли заданный \textit{концепт} \textit{выполнимым} относительно заданного \textit{TBox}?
	\item вложенность \textit{концептов}: верно ли, что один заданный \textit{концепт} вложен в другой относительно заданного \textit{TBox}?
	\item совместимость \textit{TBox}: имеет ли заданный \textit{TBox} хотя бы одну \textit{модель}?
	\item совместимость \textit{базы знаний}: имеет ли заданная \textit{пара} (\textit{TBox}, \textit{ABox}) хотя бы одну \textit{модель}?
	\end{scnitemize}

Кроме того, важное практическое значение имеют такие проблемы, как:
\begin{scnitemize}
\item \textit{классификация терминологии}: для данной \textit{терминологии} (то есть \textit{TBox}) построить \textit{таксономию} или \textit{иерархию концептов}, то есть упорядочить все \textit{атомарные концепты} по отношению вложения (отн. данного \textit{TBox}) и выдать соответствующее \textit{частично упорядоченное множество};
\item извлечение \textit{экземпляров концепта}: найти все \textit{экземпляры} заданного \textit{концепта} относительно заданной \textit{базы знаний};
\item наиболее узкий \textit{концепт} для \textit{индивида}: найти наименьший (по вложению) \textit{концепт}, \textit{экземпляром} которого является заданный \textit{индивид} относительно заданной \textit{базы знаний};
\item ответ на \textit{запрос} к \textit{базе знаний}: выдать все наборы \textit{индивидов}, которые удовлетворяют заданному \textit{запросу} относительно заданной \textit{базы знаний}.
\end{scnitemize}}

\scnrelfromlist{выводы к сегменту}{
	\scnfileitem {\textit{Дескриптивные логики} изначально создавались как способ \textit{представления знаний}, обладающий балансом между выразительной мощностью и вычислительной сложностью;};
	\scnfileitem {Сами по себе классические \textit{DL} достаточно ограничены и в основном ориентированы на \textit{представление таксономий} (\textit{иерархий}) \textit{классов}, а также описании \textit{свойств} между \textit{экземплярами} этих \textit{классов}. Фактически классические \textit{DL} не содержат \textit{средств описания свойств} самих \textit{классов} и \textit{отношений} между ними, \textit{средств структуризации баз знаний}, \textit{средств записи }нечетких, нестационарных и других \textit{знаний} с \textit{НЕ-факторами}.}}





\bigskip

\scnendstruct \scnendsegmentcomment{Дескриптивная логика}
\bigskip
\scnendcurrentsectioncomment