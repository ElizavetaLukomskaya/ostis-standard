\scseparatedfragment{Направления развития Semantic Web}
\scnheader{Направления развития Semantic Web}

\scneqfile{Несмотря на довольно очевидную выгоду от ее \textit{внедрения}, \textit{общая концепция Semantic Web} в настоящее время не реализована даже на базовом уровне. Уже в 2006 году Тим Бернерс-Ли опубликовал статью\textit{ “Semantic Web Revisited”}, в которой признал, что по ряду причин \textit{концепция} в большой степени не реализована.\newline
	Причин такой ситуации несколько:
	\begin{scnitemize}
		\item человеческий фактор и накладные расходы - описание \textit{метаданных} для какого-либо \textit{ресурса} и вообще его \textit{формализация} требует дополнительных затрат, иногда довольно существенных, но в то же время не дает очевидной прямой выгоды в ближайшей перспективе. Кроме того, многие коммерческие компании не заинтересованы в \textit{стандартизации} подобных \textit{решений} в принципе, поскольку это снизит объем выполняемой работы и востребованность таких компаний на рынке;
		\item сами по себе \textit{средства представления} и \textit{обработки знаний \textbf{Semantic Web}} достаточно сложны в использовании для неподготовленного человека;
		\item остается открытым вопрос с выбором \textit{универсальной системы понятий верхнего уровня}, которые бы гарантированно позволили описать любую \textit{информацию}; кроме того, возникает вопрос о том, возможно ли вообще выбрать такую \textit{систему понятий};
		\item добавление \textit{метаинформации} автоматически приводит к частичному дублированию уже имеющейся \textit{информации} и соответствующим расходам.
	\end{scnitemize}
}
\scnendcurrentsectioncomment