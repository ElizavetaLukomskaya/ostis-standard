\scnheader{Языки описания логических высказываний}
\scntext{особенность}{Для записи логических высказываний в языках семейства Semantic Web используется соответствующий набор понятий, определяемых соответствующими онтологиями, например в рамках семейства SWAP}
\scnaddlevel{1}
\scnnote{имеются в виду те классы высказываний, которые не накрываются стандартизированными отношениями в рамках соответствующего языка, например OWL 2}
\scnaddlevel{-1}
\scntext{особенность}{Для записи логических переменных используется механизм именования переменных, рассмотренный в разделе Нотация N3. (?x либо \_:x). На переменные могут накладываться кванторы, рассмотренные в том же разделе. Для связи между собой атомарных высказываний может использоваться механизм формул, также рассмотренный в указанном разделе}
\scnnote{Иногда для записи логических высказываний используется менее громоздкий синтаксис языка KIF, особенно в случае, когда сформированный таким образом набор высказываний выкладывается где-либо в открытом источнике как многократно используемый компонент}
\scntext{пример}{
	(subclass RailVehicle LandVehicle) /*ЖД транспорт - подкласс наземного*/\\
	(documentation RailVehicle\\
	"A Vehicle designed to move on \&\%Railways.") /*ЕЯ-формулировка*/\\
	(=>  (instance ?X RailVehicle) /*Если Х - ЖД транспорт, то */\\
	(hasPurpose ?X  /* Х предназначен для случаев когда */\\
	(exists (?EV ?SURF)\\
	(and  (instance ?RAIL Railway) /* существуют рельсы и */\\
	(instance ?EV Transportation) /* существует перемещение */\\
	(holdsDuring (WhenFn ?EV) /* которое происходит */\\
	(meetsSpatially ?X ?RAIL)))))) /* когда Х контактирует с рельсами */
}
\scnaddlevel{1}
\scniselement{префиксная запись KIF}
\scnaddlevel{-1}
\scniselement{SWRL}
\scnaddlevel{1}
\scnidtf{язык SWRL основан на объединении языков OWL DL и OWL Lite с ограниченной версией (подъязыком) языка RuleML, который разрабатывался как язык описания логических правил для различных целей и, в свою очередь, является подъязыком языка логического программирования Datalog, который, в свою очередь, является подъязыком языка Prolog}

\scnnote{SWRL изначально разрабатывался в рамках проекта по разработке языка DAML, который являлся предтечей языков описания онтологий в рамках Semantic Web}

\scntext{особенность}{SWRL дополняет OWL DL возможностью записи логических правил в человеко-читабельном (Хорновском) стиле. При этом SWRL не добавляет каких-либо принципиальных отличий в плане выразительной мощности}
\scntext{пример}{hasParent(?x1,?x2) ∧ hasBrother(?x2,?x3) ⇒ hasUncle(?x1,?x3)}
\scnaddlevel{1}
	\scnexplanation{Данная запись определяет отношение “быть дядей”}
\scnaddlevel{-1}
\scntext{пример}{Student(?x1) ⇒ Person(?x1)}
\scnaddlevel{1}
	\scnexplanation{Данная запись гласит, что каждый студент является персоной. Очевидно, что та же информация может быть записана без явного использования логических правил при помощи отношения owl:subClassOf}
\scnaddlevel{-1}
\scnnote{Тексты на SWRL с точки зрения синтаксиса могут быть записаны в тех же форматах, что и тексты OWL, например OWL XML (на основе RDF$\backslash$XML) и других}
\scnaddlevel{-1}
\scnrelto{ключевой знак}{\scncite{SWRLDescription}}