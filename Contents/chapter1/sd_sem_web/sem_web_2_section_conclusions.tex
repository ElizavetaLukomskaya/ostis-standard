\scntext{выводы к разделу}{\begin{scnitemize}
		\item В рассмотренных языках никак не рассматриваются проблемы представления не-факторов (нечеткости, нестационарности, недостоверности информации)
		\item Возникают проблемы с формальной трактовкой некоторых языковых средств, например, не совсем понятно, как трактуется формула в N3.  Как следствие, механизм формул и вообще средства структуризации баз знаний используются достаточно редко
		\item Не существует простой и однозначной формы записи некоторых языковых средств, таких как формулы и коллекции в N3 на нижнем уровне (например, средствами чистого RDF), что также препятствует их широкому использованию. Например, по этой причине такие конструкции приходится явно учитывать при реализации хранилищ и средств доступа к ним
		\item Несмотря на изначально простой абстрактный синтаксис RDF и языков на его основе, в реальной практике работа как правило ведется на уровне исходного текста, записанного в какой-то из форм представления, например RDF$\backslash$XML. Разработчики каждой конкретной системы сами определяют, каким образом эти тексты преобразуются во внутреннее представление в системе. Соответственно принципы обработки этой информации в общем случае не стандартизируются (см. раздел Обработка информации)
		\item Отсутствует какая-либо строгая и одновременно простая формальная база для представления информации (ядро, инвариант представления), которая была бы универсальной и на основе которой строились бы все остальные средства. В данной роли фактически выступает RDF, но он в полной мере не отвечает ни требованию универсальности, ни требованию формальности
		\item Не выделено каких-либо базовых отношений, которые бы записывались на уровне синтаксиса базового языка. Все отношения приходится указывать явно
		\item Таким образом, рассмотренные стандарты позволяют структурировать информационное пространство, облегчить поиск нужной информации, обеспечить согласованность описания ресурсов различными авторами, т.е. успешно решают те задачи, для которых они создавались, но не могут рассматриваться как универсальные языки представления информации любого рода в базах знаний
	\end{scnitemize}
}