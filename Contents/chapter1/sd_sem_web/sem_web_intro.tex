\scseparatedfragment{Введение в Semantic Web}
\scnheader{Введение в Semantic Web}

	\scneqfile{В марте 1989 года Тим Бернерс-Ли предложил концепцию распределенной информационной системы с целью "объединения знаний человечества", которую он назвал \textit{"Всемирной паутиной"} (\textit{World Wide Web} - \textit{WWW}). Для её создания он объединил две существующие технологии - \textit{технологию применения IP-протоколов для передачи данных} и \textit{технологию гипертекста} (\textit{Hypertext Technology}).
	Основными элементами технологии \textit{WWW} являются :
	\begin{scnitemize}
		\item \textit{язык гипертекстовой разметки документов} (\textit{Hyper Text Markup Language} - \textit{HTML});
		\item \textit{протокол обмена гипертекстовой информацией} (\textit{Hyper Text Transfer Protocol} - \textit{HTTP});
		\item \textit{универсальный способ адресации ресурсов в сети} (\textit{Universal Resource Identifier} - \textit{URI}, и \textit{Universal Resource Locator} - \textit{URL});
		\item \textit{система доменных имен} (\textit{Domain Name System} - \textit{DNS});
		\item \textit{универсальный интерфейс шлюзов} (\textit{Common Gateway Interface} - \textit{CGI}), добавленный позже сотрудниками \textit{Национального Центра Суперкомпьютерных Приложений} (\textit{National Center for Supercomputing Applications} - \textit{NCSA});
		\item \textit{расширяемый язык разметки} (\textit{eXtensible Markup Language} - \textit{XML}), рекомендованный \textit{Консорциумом Всемирной паутины}.
	\end{scnitemize}
	
	Основная цель \textbf{\textit{Semantic Web}} создать надстройку над существующей \textit{Всемирной паутиной}, позволяющую сделать размещаемую в \textit{Интернете} информацию пригодной для машинной обработки, а именно для автоматического анализа, синтеза выводов и преобразования как самих данных, так и сделанных на их основе заключений в различные представления, полезные на практике. 
	
	В частности, такой подход позволит повысить релевантность предоставляемой пользователю информации, обеспечить предоставление только необходимой части информации, облегчить поиск информации и повысить эффективность ее использования.
	
	Важно отметить, что концепции \textit{Web 2.0} и \textit{Web 3.0}, описывающие эволюцию интернет-технологий, не связаны непосредственно с \textit{\textbf{Semantic Web}}, более того, автор концепции \textit{Web 2.0} Тим О’Рейли неоднократно выступал с критикой отождествления \textit{\textbf{Semantic Web}} и \textit{Web 3.0}.}
\bigskip
\scnendcurrentsectioncomment

