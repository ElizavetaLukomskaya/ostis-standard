\scnheader{Ранние языки представления информации}
\scniselement{KL-ONE}
\scnaddlevel{1}
	\scnidtf{система языков, разработанная в 1980ых и интегрирующая идеи семантических сетей и фреймов (понятие класса, подкласса, наследования и т.д.), на основе которых были построены одни из первых средств прямого вывода на онтологиях (на основе отношения класс-подкласс)}
\scnaddlevel{-1}

\scniselement{LOOM}
\scnaddlevel{1}
	\scnidtf{язык со строгой формальной семантикой, разработанный в 1990ых на основе \textit{KL-ONE}, конструкциям которого могут быть поставлены в соответствие либо теоретико-множественные выражения, либо выражения логики предикатов первого порядка}
\scnaddlevel{-1}

\scniselement{KIF}
\scnaddlevel{1}
	\scnidtf{Knowledge Interchange Format}
	\scnidtf{язык, схожий с \textit{LOOM} и другими \textit{KL-ONE} языками, но предназначенный в первую очередь для обмена знаниями между компьютерными системами}
	\scntext{особенность}{Тексты \textit{KIF} могут быть протранслированы в \textit{RDF} и наоборот}
	\scnreltolist{ключевой знак}{\scncite{KIFa};\scncite{KIFb}}
\scnaddlevel{-1}

\scniselement{KQML}
\scnaddlevel{1}
	\scnidtf{Knowledge Query and Manipulation Language}
	\scnidtf{язык, предназначенный для обмена сообщениями между агентами, в настоящее время вытеснен стандартом ACL (Agent Communication Language) от FIPA}
\scnaddlevel{-1}