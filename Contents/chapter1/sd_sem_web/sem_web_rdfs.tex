\scnheader{RDFS}
\scnidtf{RDF Schema}
\scnidtf{RDF/S}
\scnidtf{RDF-S}
\scnidtf{RDF(S)}
\scnidtf{RDF Schema}
\scnidtf{набор классов и свойств для модели представления знаний RDF, составляющий основу для описания онтологий с использованием расширенного RDF-словаря}

\scnnote{RDFS использует кодирование в виде RDF, то есть позволяет описать сами RDF-триплеты подобно описаниям других RDF-ресурсов}

\scntext{особенность}{RDFS является базовых словарем, на основе которого строится большинство других словарей, включая язык OWL}

\scnrelfrom{классы}{Классы RDFS\\
	\scniselement{rdfs:Resource\\
		\scnaddlevel{1}
			\scnidtf{класс, включающий все ресурсы, то есть, всё, что описывает RDF}
		\scnaddlevel{-1}
	}
	\scniselement{rdfs:Class\\
		\scnaddlevel{1}
			\scnidtf{описывает что ресурс является классом (типом) ресурса. Для отнесения ресурса к типу используется свойство \textit{rdf:type}
			}
		\scnaddlevel{-1}
	}
	\scniselement{rdfs:Literal\\
		\scnaddlevel{1}
			\scnidtf{обозначает литерал (строка, смысл которой уточняется отдельно при помощи IRI типа данных}
			\scnnote{Литералы могут быть простыми (plain, например, число) или иметь некоторый тип}
		\scnaddlevel{-1}
	}
	\scniselement{rdfs:Datatype\\
		\scnaddlevel{1}
			\scnidtf{класс типов данных}
			\scntext{особенность}{Является одновременно и подклассом \textit{rdfs:Class}, и экземпляром из \textit{rdfs:Class}. Каждый экземпляр класса \textit{rdfs:Datatype} является подклассом \textit{rdfs:Literal}}	
		\scnaddlevel{-1}
	}
	\scniselement{rdf:XMLLiteral\\
		\scnaddlevel{1}
			\scnidtf{класс XML-литералов, является экземпляром \textit{rdfs:Datatype}
			}
		\scnaddlevel{-1}
	}
	\scniselement{rdf:Property\\
		\scnaddlevel{1}
			\scnidtf{класс свойств (отношений)}
		\scnaddlevel{-1}
	}
}
\scnrelfrom{свойства}{Свойства RDFS\\
	\scnidtf{описывают отношения между ресурсами-субъектами и ресурсами-объектами и являются экземплярами класса \textit{rdf:Property}, и выступают в качестве предикатов в RDF-триплетах}
	\scnrelfrom{включение}{метасвойства RDFS}\\
	\scnaddlevel{1}
	\scntext{особенность}{могут описывать свойства самих предикатов}
	\scniselement{rdfs:domain\\
		\scnaddlevel{1}
			\scnidtf{объявляет класс субъекта (первый домен отношения)}
			\scntext{пример}{ex:employer rdfs:domain foaf:Person\\
				ex:employer rdfs:range foaf:Organization\\
				ex:John ex:employer ex:CompanyX
			}
			\scnaddlevel{1}
				\scnnote{В этом примере пространство имен ex: означает, что это пример}
			\scnaddlevel{-1}
		\scnaddlevel{-1}
	}
	\scniselement{rdfs:range\\
		\scnaddlevel{1}
			\scnidtf{объявляет класс или тип данных объекта (второй домен отношения)}
			\scntext{пример}{ex:employer rdfs:domain foaf:Person\\
				ex:employer rdfs:range foaf:Organization\\
				ex:John ex:employer ex:CompanyX
			}
			\scnaddlevel{1}
				\scnnote{В этом примере пространство имен ex: означает, что это пример}
			\scnaddlevel{-1}
		\scnaddlevel{-1}
	}
	\scniselement{rdf:type\\
		\scnaddlevel{1}
			\scnidtf{декларирует принадлежность ресурса некоторому классу, то есть, тот факт, что ресурс является экземпляром класса}
			\scntext{пример}{ex:John ex:type foaf:Person}
		\scnaddlevel{-1}
	}
	\scniselement{rdfs:subClassOf\\
		\scnaddlevel{1}
			\scnidtf{свойство, позволяющее описать иерархию классов}
			\scntext{пример}{foaf:Person rdfs:subClassOf foaf:Agent}
		\scnaddlevel{-1}
	}
	\scniselement{rdfs:subPropertyOf\\
		\scnaddlevel{1}
			\scnidtf{свойство, которое утверждает, что все ресурсы, связанные некоторым подсвойством (subproperty), связаны также и свойством}
		\scnaddlevel{-1}
	}
	\scniselement{rdfs:label
		\scnaddlevel{1}
			\scnidtf{задает удобные для человека имя и описание ресурса (на естественном языке)}
		\scnaddlevel{-1}
	}
	\scniselement{rdfs:comment
		\scnaddlevel{1}
			\scnidtf{задает удобные для человека имя и описание ресурса (на естественном языке)}
		\scnaddlevel{-1}
	}
	\scnaddlevel{-1}
}
\scnrelfrom{вспомогательные свойства}{Вспомогательные свойства RDFS\\
	\scnnote{К вспомогательным относятся свойства, не имеющие семантики вывода, т.е. семантика этих свойств зависит от конкретных приложений, интерпретирующих RDFS}
	\scniselement{rdfs:seeAlso\\
		\scnaddlevel{1}
			\scnidtf{указывает ресурс, который может служить источников дополнительной информации о ресурсе-субъекте}
		\scnaddlevel{-1}
	}
	\scniselement{rdfs:isDefinedBy\\
		\scnaddlevel{1}
			\scnidtf{указывает на ресурс (например, RDF-словарь), который описывает ресурс-субъект}
		\scnaddlevel{-1}
	}
}
\scnrelto{ключевой знак}{\scncite{RDFSDescription}}