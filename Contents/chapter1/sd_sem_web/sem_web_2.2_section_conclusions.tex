\scntext{выводы к разделу}{
	\begin{scnitemize}
		\item Одной из основных задач \textit{URI} и других стандартов идентификации является возможность отличать сущности, имеющие одинаковое название, но возможно, разную трактовку в разных онтологиях (пространствах имен). Данная проблема действительно важна и она решается существующими стандартами;
		\item Вопрос с решением аналогичной проблемы в \textit{OSTIS} активно не поднимался, нужно подумать, актуальна ли она для нас. У нас нет жесткой привязки к именам и в принципе нет трагедии в том, что два разных \textit{sc-элемента} будут иметь одинаковые имена. Проблема возникнет в ситуации, когда мы захотим объединить две базы знаний, где есть сущности с разной семантикой, но одинаковыми именами;
		\item Отправной точкой при разработке стандартов идентификации ресурсов являлась необходимость описания ресурсов в глобальной сети, а не построение модели мира вообще. В связи с этим терминология, используемая при определении идентификаторов, их синтаксис и т.д. “заточены” под концепцию всемирной паутины, однако их пытаются применять как основу для описания любых предметных областей, что не всегда получается понятно и логично. Например, при описании геометрии придется оперировать понятиями путь, запрос, ресурс и т.д., что довольно странно;
		\item Существуют другие недостатки, в частности, создатель \textit{URI}, Тим Бернерс-Ли, говорил, что система доменных имён, лежащая в основе \textit{URL}, — плохое решение, навязывающее ресурсам иерархическую архитектуру, мало подходящую для гипертекстового веба.
	\end{scnitemize}
}