\scnheader{Примеры широко используемых словарей$\backslash$онтологий}

\scniselement{Дублинское ядро}
\scnaddlevel{1}
	\scnidtf{Dublin Core}
	\scnidtf{широко используемый словарь, предназначенный для унификации метаданных для описания широчайшего диапазона ресурсов}
	\scnrelfrom{разбиение}{Разделение по уровням}
	\scnaddlevel{1}
		\scneqtoset{простой\\
			\scnaddlevel{1}
				\scnidtf{неквалифицированный}
				\scnidtf{simple}
				\scniselement{Title}
				\scnaddlevel{1}
					\scnidtf{название}
				\scnaddlevel{-1}
				\scniselement{Creator}
				\scnaddlevel{1}
					\scnidtf{создатель}
				\scnaddlevel{-1}
				\scniselement{Subject}
				\scnaddlevel{1}
					\scnidtf{тема}
				\scnaddlevel{-1}
				\scniselement{Description}
				\scnaddlevel{1}
					\scnidtf{описание}
				\scnaddlevel{-1}
				\scniselement{Publisher}
				\scnaddlevel{1}
					\scnidtf{издатель}
				\scnaddlevel{-1}
				\scniselement{Contributor}
				\scnaddlevel{1}
					\scnidtf{внёсший вклад}
				\scnaddlevel{-1}
				\scniselement{Date}
				\scnaddlevel{1}
					\scnidtf{дата}
				\scnaddlevel{-1}
				\scniselement{Type}
				\scnaddlevel{1}
					\scnidtf{тип}
				\scnaddlevel{-1}
				\scniselement{Format}
				\scnaddlevel{1}
					\scnidtf{формат документа}
				\scnaddlevel{-1}
				\scniselement{Identifier}
				\scnaddlevel{1}
					\scnidtf{идентификатор}
				\scnaddlevel{-1}
				\scniselement{Source}
				\scnaddlevel{1}
					\scnidtf{источник}
				\scnaddlevel{-1}
				\scniselement{Language}
				\scnaddlevel{1}
					\scnidtf{язык}
				\scnaddlevel{-1}
				\scniselement{Relation}
				\scnaddlevel{1}
					\scnidtf{отношения}
				\scnaddlevel{-1}
				\scniselement{Coverage}
				\scnaddlevel{1}
					\scnidtf{покрытие}
				\scnaddlevel{-1}
				\scniselement{Rights}
				\scnaddlevel{1}
					\scnidtf{авторские права}
				\scnaddlevel{-1}
			\scnaddlevel{-1}
			;компетентный\\
			\scnaddlevel{1}
				\scnidtf{квалифицированный}
				\scnidtf{qualified}
				\scniselement{Audience}
				\scnaddlevel{1}
					\scnidtf{аудитория (зрители)}
				\scnaddlevel{-1}
				\scniselement{Provenance}
				\scnaddlevel{1}
					\scnidtf{происхождение}
				\scnaddlevel{-1}
				\scniselement{RightsHolder}
				\scnaddlevel{1}
					\scnidtf{правообладатель}
				\scnaddlevel{-1}
			\scnaddlevel{-1}
		}
	\scnaddlevel{-1}
	\scnnote{В России с 1 июля 2011 года действует ГОСТ «Национальный стандарт Российской Федерации. Система стандартов по информации, библиотечному и издательскому делу. Набор элементов метаданных „Дублинское ядро“»}
	\scnrelto{ключевой знак}{\scncite{DublinCore}}
\scnaddlevel{-1}

\scniselement{Schema.org}
\scnaddlevel{1}
	\scnidtf{совместная инициатива по разработке единой схемы для семантической разметки в HTML5}
	\scnnote{Инициатива была запущена второго июня 2011 года создателями крупнейших поисковых систем — компаниями Google, Yahoo! и Microsoft, а с осени 2011 года к ней присоединился Яндекс}
	\scntext{особенность}{В рамках проекта разработан набор словарей, описанных на сайте schema.org и включающих термины из всевозможных предметных областей, которые могут быть полезны при описании каких-либо интернет-ресурсов}
	\scnaddlevel{1}
		\scnexplanation{По сути, данное описание представляет собой одну из онтологий верхнего уровня}
	\scnaddlevel{-1}
	\scniselement{Thing}
	\scnaddlevel{1}
		\scnidtf{базовое (верхнее) понятие, от которого наследуются все остальные}
		\scnsubset{Action}
		\scnsubset{CreativeWork}
		\scnsubset{Event}
		\scnsubset{Intangible}
		\scnaddlevel{1}
			\scnidtf{нематериальный объект}
		\scnaddlevel{-1}
		\scnsubset{MedicalEntity}
		\scnsubset{Organization}
		\scnsubset{Person}
		\scnsubset{Place}
		\scnsubset{Product}
	\scnaddlevel{-1}
	\scnrelfromset{состав понятия}{
		\scnfileitem{естественно-языковое пояснение};
		\scnfileitem{перечень свойств экземпляров в виде <имя свойства> - <тип значения свойства> - <пояснение свойства>};
		\scnfileitem{перечень свойств других понятий, значением которых может быть экземпляр данного класса, в том же виде};
		\scnfileitem{список более частных понятий}
	}
	\scnrelto{ключевой знак;~пример}{\scncite{Person}}
	\scnrelto{ключевой знак}{\scncite{SchemaOrg}}
\scnaddlevel{-1}

\scniselement{FOAF}
\scnaddlevel{1}
	\scnidtf{Friend of a Friend}
	\scnidtf{проект по созданию модели домашних страниц различных персон и социальных сетей}
	\scntext{примеры терминов}{foaf:Person\\
		foaf:knows\\
		foaf:age\\
		foaf:homepage}
	\scnrelto{ключевой знак}{\scncite{FOAF}}
\scnaddlevel{-1}

\scniselement{DOAP}
\scnaddlevel{1}
	\scnidtf{Description of a Project}
	\scnidtf{RDF схема и набор инструментов для описания проектов разработки программного обеспечения}
	\scntext{примеры терминов}{doap:programming-language\\
		doap:Project\\
		doap:license}
	\scnrelto{ключевой знак}{\scncite{DOAP}}
\scnaddlevel{-1}

\scniselement{SKOS}
\scnaddlevel{1}
	\scnidtf{Simple Knowledge Organization System}
	\scnidtf{разработанная консорциумом W3 модель организации знаний для семантической паутины, призванная облегчить взаимодействие различных информационных систем за счёт стандартизации тезаурусов, систем классификации, таксономий, фолксономий и других видов нормализации лексики}
	\scntext{особенность}{С момента выхода второй версии в 2008 году SKOS получил достаточно широкое распространение. Многие большие тезаурусы были опубликованы в виде SKOS, в том числе предметный указатель Библиотеки Конгресса, а также такие тезаурусы как EUROVOC, AGROVOC}
	\scntext{примеры терминов}{skos:broader (более общий термин)\\
	skos:narrower (более частный термин)\\
	skos:related (связанный термин)\\
	skos:exactMatch, skos:narrowMatch, skos:broadMatch, skos:closeMatch (отношения для описания близости терминов из разных словарей)}
	\scnrelto{ключевой знак}{\scncite{SKOS}}
\scnaddlevel{-1}

\scniselement{SIOC}
\scnaddlevel{1}
	\scnidtf{Semantically-Interlinked Online Communities}
	\scnidtf{технология, которая предоставляет методы для связи ресурсов, позволяющих проводить различные обсуждения, таких как блоги, форумы, списки рассылки и т.д., друг с другом}
	\scntext{особенность}{Включает соответствующую онтологию, хранилища и семейство средств обработки такого рода информации}
	\scnrelto{ключевой знак}{\scncite{SIOCa};\scncite{SIOCb}}
\scnaddlevel{-1}

\scniselement{RSS}
\scnaddlevel{1}
	\scnidtf{Rich Site Summary}
	\scnidtf{семейство XML-форматов, предназначенных для описания лент новостей, анонсов статей, изменений в блогах и т. п}
	\scntext{особенность}{В версиях 0.9 и 1.0 расшифровывалось как RDF Site Summary и частично использовало RDF$\backslash$XML}
\scnaddlevel{-1}

\scniselement{SWAP}
\scnaddlevel{1}
	\scnidtf{Semantic Web Application Platform}
	\scnexplanation{Отдельного внимания заслуживает платформа SWAP, которая официально не утверждена W3C, но широко используется сообществом}
	\scnrelfromset{включение}{Набор базовых онтологий\\
		\scnaddlevel{1}
			\scnidtf{Набор базовых словарей}
			\scniselement{Logic}
			\scnaddlevel{1}
				\scnidtf{основные логические операции, понятия истина и ложь, кванторы}
			\scnaddlevel{-1}
			\scniselement{String}
			\scnaddlevel{1}
				\scnidtf{базовые понятия из области обработки строк - конкатенация, сравнение и т. д.}
			\scnaddlevel{-1}
			\scniselement{Math}
			\scnaddlevel{1}
				\scnidtf{базовые понятия из математики - арифметика, тригонометрия}
			\scnaddlevel{-1}
			\scniselement{Crypto}
			\scnaddlevel{1}
				\scnidtf{несколько базовых понятий из области защиты информации}
			\scnaddlevel{-1}
			\scniselement{OS}
			\scnaddlevel{1}
				\scnidtf{несколько понятий для связи с современными ОС}
			\scnaddlevel{-1}
		\scnaddlevel{-1}
		;Программное средство Cwm\\
		\scnaddlevel{1}
			\scnidtf{Программное средство Closed Word Machine}
			\scnidtf{представляет собой в первую очередь ризонер, способный осуществлять прямой логический вывод на основе логических утверждений, записанных в N3, а также содержит некоторые дополнительные утилиты, такие как конвертеры для разных форматов RDF, средства слияния файлов и другие}
			\scnrelto{ключевой знак}{\scncite{CWM}}
		\scnaddlevel{-1}
	}
	\scnrelto{ключевой знак}{\scncite{SWAP}}
\scnaddlevel{-1}