\begin{SCn}

\input{Contents/toc}

\newpage

\scseparatedfragment{Титульная спецификация Стандарта OSTIS}

\begin{SCn}

\scnsectionheader{\currentname}

\scnstartsubstruct

\scnrelto{титульная спецификация}{Стандарт OSTIS}
\scnaddlevel{1}
	\scnrelfrom{оглавление}{Оглавление Стандарта OSTIS}
	\scnrelfrom{общая структура}{Общая Структура Стандарта OSTIS}
	\scnrelfrom{система ключевых знаков}{Система ключевых знаков Стандарта OSTIS}
	\scnrelfrom{редакционная коллегия}{Редакционная коллегия Стандарта OSTIS}
	\scnrelfrom{авторский коллектив}{Авторский коллектив Стандарта OSTIS}
	\scnrelfrom{направления развития}{Направления развития Стандарта OSTIS}
	\scnrelfrom{правила построения}{Правила построения Стандарта OSTIS}
	\scnaddlevel{1}
		\scnidtf{правила построения*(Стандарт OSTIS)}
		\scnaddlevel{1}
			\scniselement{sc-выражение}
		\scnaddlevel{-1}
	\scnaddlevel{-1}
	\scnrelfrom{правила организации развития}{Правила организации развития Стандарта OSTIS}
	\scnaddlevel{1}	
		\scnrelfromset{декомпозиция}{Правила организации развития исходного текста Стандарта OSTIS;Правила организации развития Стандарта OSTIS на уровне его внутреннего представления в памяти Метасистемы IMS.ostis}
	\scnaddlevel{-1}
\scnaddlevel{-1}

\scnheader{Стандарт OSTIS-2021}
\scnidtf{Издание Документации Технологии OSTIS-2021}
\scnidtf{Первое издание (публикация) Внешнего представления Документации Технологии OSTIS в виде книги}
\scniselement{публикация}
\scnaddlevel{1}
\scnidtf{библиографический источник}
\scnaddlevel{-1}
\scniselement{официальная версия Стандарта OSTIS}
\scniselement{бумажное издание}
\scniselement{научное издание}
\scnrelfrom{рекомендация издания}{Совет БГУИР}
\scnrelfromset{рецензенты}{Курбацкий А.Н.; Дудкин А.А.}
\scnrelfrom{издательство}{Бестпринт}
\scniselement{\scnstartsetlocal\scnendstructlocal}
\scnaddlevel{1}
\scniselement{УДК}
\scnaddlevel{1}
\scniselement{параметр}
\scnaddlevel{-1}
\scnrelfrom{Индекс УДК}{004.8}
\scnaddlevel{-1}
\scnidtftext{ISBN}{978-985-7267-13-2}

\scnheader{Стандарт OSTIS-2022}
\scnidtf{Издание Документации Технологии OSTIS-2022}
\scnidtf{Второе издание (публикация) Внешнего представления Документации Технологии OSTIS в виде книги}
\scniselement{публикация}
\scniselement{официальная версия Стандарта OSTIS}
\scniselement{бумажное издание}
\scniselement{научное издание}

\bigskip
\scnendstruct \scninlinesourcecommentpar{Завершили Титульную спецификацию \textit{Стандарта OSTIS}}

\end{SCn}

\newpage

\scseparatedfragment{Титульная спецификация второго издания Стандарта OSTIS}

\begin{SCn}
	
\scnsectionheader{\currentname}
\scnstartsubstruct

\scnidtf{Титульная спецификация Стандарта OSTIS-2022}

\scnrelto{титульная спецификация}{Стандарт OSTIS-2022}

\scnaddlevel{1}	

\scnidtf{Второе издание стандарта OSTIS}
\scnrelfrom{предисловие}{Предисловие ко второму изданию Стандарта OSTIS}



\scnaddlevel{-1}	
\scnendstruct

\end{SCn}

\newpage

\scseparatedfragment[\scnidtf{Предисловие Стандарта OSTIS-2021}]{Предисловие к первому изданию Стандарта OSTIS}

\begin{SCn}
	
	\scnsectionheader{\currentname}
	
	\scneqfile{В настоящее время уровень требований, предъявляемых к комплексу \textit{технологий Искусственного интеллекта} существенно повысился -- возникла необходимость разработки \textit{компьютерных систем}, которые не только обладают высоким уровнем \textit{интеллекта}, но и обладают \textit{семантической совместимостью}, взаимопониманием, способностью координировать свою деятельность с другими системами при коллективном решении сложных "внештатных"{} задач. Очевидно, что эти требования предполагают существенное развитие  \textit{стандартов интеллектуальных компьютерных систем}. Важнейшая особенность \textit{стандарта интеллектуальных компьютерных систем}, обеспечивающего их \textit{семантическую совместимость} и взаимопонимание, заключается в том, что для этого \textit{интеллектуальные компьютерные системы} должны использовать:
		\begin{scnitemize}
			\item   один и тот же язык внутреннего представления знаний;
			\item   один и тот же язык их коммуникации;
			\item   одну и ту же \textit{систему понятий};
			\item   обладать достаточно большим количеством  общих (одинаковых) знаний.
		\end{scnitemize}
		
		Следовательно, разработка \textit{стандарта интеллектуальных компьютерных систем} в той части этого \textit{стандарта}, которая связана с выделением и формализацией общих (одинаковых) \textit{знаний}, хранимых в \textit{памяти интеллектуальной компьютерной системы} и необходимых для обеспечения их взаимопонимания, фактически осуществляет разработку достаточно большой  одинаковой части всех \textit{интеллектуальных компьютерных систем} (и не только прикладных),  что существенно  сокращает сроки их разработки.
		
		К этому можно добавить возможность и целесообразность  одинаковой  для всех \textit{интеллектуальных компьютерных систем} реализации целого ряда их способностей: 
		\begin{scnitemize}
			\item способности \uline{понимать} информацию, которой они обмениваются между собой,
			\item способности  \uline{договариваться} и \uline{координировать} свои действия при коллективном решении сложных интеллектуальных задач в условиях нештатных (аномальных, нестандартных, непредусмотренных) ситуаций,
			\item способности \uline{принимать решения} на основе их глубокого  обоснования,
			\item способности \uline{обучаться} и многие другие способности, обеспечивающие необходимый уровень интеллекта разрабатываемых интеллектуальных компьютерных систем.
		\end{scnitemize}
		
		Данная монография является первым этапом на пути комплексного решения указанных выше проблем.
		Предназначена она одновременно:
		\begin{scnitemize}
			\item для студентов, магистрантов и аспирантов, обучающихся по специальности ``\textit{Искусственный интеллект}'';
			\item для разработчиков прикладных \textit{интеллектуальных компьютерных систем};
			\item для разработчиков технологий проектирования и производства \textit{интеллектуальных компьютерных систем};
			\item для научных работников, создающих новые \textit{модели} и \textit{методы} для решения \textit{интеллектуальных задач}.
		\end{scnitemize}
		
		Данная монография сочетает:
		\begin{scnitemize}
			\item строгую формализацию представляемой \textit{информации} и её доступность (возможность первичного её понимания без предварительного изучения используемого \textit{\uline{формального} языка});
			\item традиционную ("пассивную"{}) форму представления материала (в "электронном"{} и "бумажном"{} виде) с "активной"{} формой в виде \textit{интеллектуальной справочной системы}, когда \textit{компьютерная система} не только обеспечивает оперативное \textit{редактирование информации}, но и помогает пользователям  быстрее и  качественнее усваивать эту \textit{информацию} (за счет возможности отвечать на широкий спектр вопросов и учитывать индивидуальные особенности, потребности и интересы \textit{пользователей}). 
		\end{scnitemize}
		
		Область \textit{Искусственного интеллекта} сочетает в себе как  научно-исследовательский аспект и   создание \textit{технологий} разработки \textit{интеллектуальных компьютерных систем}, а так и непосредственно разработку самих \textit{интеллектуальных компьютерных систем}. Эта область развивается настолько быстрыми темпами, что за время обучения студентов и магистрантов ситуация в области \textit{Искусственного интеллекта} меняется существенно, поэтому подготовка специалистов в этой области требует особого подхода, учитывающего высокий уровень сложности этой \textit{научно-технической области}, а также быстрые темпы развития теории \textit{интеллектуальных компьютерных систем}, технологий их проектирования, а также непосредственно практики разработки конкретных прикладных \textit{интеллектуальных компьютерных систем}.
		
		Если специалист в области \textit{Искусственного интеллекта} не будет постоянно ориентироваться в тенденциях развития каждого из этих направлений развития работ в этой области, то он быстро перестанет быть конкурентноспособным. Это значит, что специалист в области \textit{Искусственного интеллекта} должен быть в достаточной степени и ученым, и создателем технологий следующего поколения, и разработчиком конкретных приложений.
		
		Таким образом, подготовку специалистов в области \textit{Искусственного интеллекта} необходимо ориентировать не на конкретное состояние науки, технологии и практики в этой области, а на перманентный процесс эволюции всех этих направлений.
		
		Сформировать у студентов и магистрантов реальные навыки в области \textit{Искусственного интеллекта} можно только путем поэтапного и непосредственного их включения в  реальную деятельность в этой области (и в \textit{научно-исследовательскую деятельность}, и в развитие \textit{технологий искусственного интеллекта}, и в разработку \textit{прикладных интеллектуальных компьютерных систем} на основе текущего состояния соответствующих технологий). Но для этого необходимо создать соответствующую научно-исследовательскую и инженерную инфраструктуру.
		
		\textit{Научно-исследовательская деятельность  в области Искусственного интеллекта} предполагает исследование феномена \textit{интеллекта} и создание принципиально новых подходов (моделей и методов) к решению \textit{интеллектуальных задач} и к разработке принципов организации соответствующих \textit{компьютерных систем}.
		
		\uline{\textit{Развитие технологий Искусственного интеллекта}} включает в себя:
		\begin{scnitemize}
			\item разработку стандарта интеллектуальных компьютерных систем, соответствующего текущему состоянию \textit{технологий искусственного интеллекта};
			\item разработку методов, средств проектирования и реализации интеллектуальных компьютерных систем.
		\end{scnitemize}
		
		\textit{Разработка прикладных интеллектуальных компьютерных} систем  предполагает грамотное применение соответствующих \textit{технологий}.
		
		
		\textit{Учебную деятельность в области Искусственного интеллекта} необходимо ориентировать не только на формирование навыков разработки \textit{прикладных интеллектуальных компьютерных систем} по заданной \textit{технологии}, но и на формирование навыков перманентного совершенствования как непосредственно самих \textit{прикладных интеллектуальных компьютерных систем}, так и \textit{технологий} их разработки, а также на изучение принципов (моделей и методов) решения \textit{интеллектуальных задач} и организации \textit{интеллектуальных систем}.
		
		
		Данная монография рассматривается нами как первый выпуск целой серии коллективных монографий, которые будут представлять последующие версии \textbf{\textit{Стандарта Технологии OSTIS}} (Open Semantic Technology for Intelligent Systems -- Стандарта \textit{технологии}, ориентированной на разработку \textit{семантически совместимых интеллектуальных компьютерных систем}). При этом предполагается существенное расширение авторского коллектива и организация всей работы на развитие \textit{Стандарта Технологии OSTIS} как \textit{открытого проекта}, целью которого является коллективное совершенствование \textit{базы знаний}, посвященной детальному описанию этого \textit{стандарта}.
		
		При этом при подготовке даже данного издания текущей версии \textit{Стандарта Технологии OSTIS} мы приобрели хороший опыт организации коллективной деятельности такого рода, привлекая к этой работе целый ряд аспирантов, магистрантов и студентов, а также сотрудников других организаций.
		
		Вклад некоторых из них в ряд разделов монографии позволил включить их в число \textit{соавторов} этих разделов, что отражено непосредственно в тексте монографии.
		
		Авторы выражают благодарность:
		
		\begin{scnitemize}
			\item Cотрудникам кафедры Интеллектуальных информационных технологий Белорусского государственного университета информатики и радиоэлектроники и кафедры Интеллектуальных информационных технологий Брестского государственного технического университета, а также сотрудникам ОАО <<Савушкин продукт>>;
			\item студентам кафедры Интеллектуальных информационных технологий Белорусского государственного университета информатики и радиоэлектроники Банцевич К.А., Бутрину С.В., Василевской А.П., Меньковой Е.А., Жмырко А.В., Григорьевой И.В., Загорскому А.Г., Марковцу В.С., Киневичу Т.О. за оказание технической помощи при подготовке текста к печати;
			\item ООО <<Интелиджент семантик системс>> и его генеральному директору Т. Грюневальду за финансовую поддержку работ по развитию \textit{Технологии OSTIS}, а также финансовую поддержку издания \textit{Стандарта OSTIS};
			\item Рецензентам -- д-ру техн. наук, профессору Александру Николаевичу Курбацкому и д-ру техн. наук, профессору Александру Арсентьевичу Дудкину;
			\item Коллегам из Советской (ныне Российской) Ассоциации Искусственного интеллекта и коллегам из Белорусского объединения специалистов в области искусственного интеллекта;
			\item Членам Программного Комитета ежегодных \textit{конференций OSTIS}, а также всем участникам этих конференций за плодотворное и конструктивное обсуждение направлений развития семантических технологий и \textit{Технологии OSTIS} в частности.
	\end{scnitemize}}
	
	\newpage
	
\end{SCn}

\newpage



\bigskip
\scnfragmentcaption

\scnheader{Пояснения к оглавлению Стандарта OSTIS и к некоторым разделам этого Стандарта}

\scnstartsubstruct

\scnheader{Спецификация второго издания Стандарта OSTIS}
\scnidtf{Спецификация второй официальной версии Стандарта OSTIS}
\scnidtf{Спецификация Стандарта OSTIS-2022}

\scnheader{Анализ методологических проблем современного состояния работ в области Искусственного интеллекта}
\scnidtf{Актуальность Технологии OSTIS}
\scnidtf{Современные требования, предъявляемые к деятельности в области Искусственного интеллекта  к интеллектуальным компьютерным системам следующего поколения -- конвергенция, глубокая ("бесшовная"{}) интеграция, высокий уровень обучаемости (гибкости, стратифицированности, рефлексивности), высокий уровень социализации (взаимопонимания, договороспособности, способности координировать свои действия с другими субъектами), стандартизация}

\scnheader{Введение в описание внутреннего языка ostis-систем}
\scnidtf{Введение в SC-code (Semantic Computer Code)}

\scnheader{Предметная область и онтология внешних идентификаторов знаков, входящих в информационные конструкции внутреннего языка ostis-систем}
\scnidtf{Предметная область и онтология sc-идентификаторов}

\scnheader{Введение в описание языка графического представления информационных конструкций, хранимых в памяти ostis-систем}
\scnidtf{Введение в SCg-code (Semantic Code graphical)}

\scnheader{Введение в описание языка линейного представления информационных конструкций, хранимых в памяти ostis-систем}
\scnidtf{Введение в SCs-code (Semantic Code string)}

\scnheader{Введение в описание языка форматирования линейного представления информационных конструкций, хранимых в памяти ostis-систем}
\scnidtf{Введение в SCn-code (Semantic Code natural)}

\scnheader{Предметная область и онтология кибернетических систем}
\scnidtf{Предпосылки создания компьютерных систем нового поколения}

\scnheader{Предметная область и онтология компьютерных систем}
\scnidtf{Этапы эволюции (повышения качества) компьютерных систем -- эволюции памяти, информации, хранимой в памяти, решателей задач, интерфейсов}

\scnheader{Предметная область и онтология интеллектуальных компьютерных систем}
\scnidtf{Этапы эволюции (повышения качества) интеллектуальных компьютерных систем и проблемы дальнейшей их эволюции}

\scnheader{Предметная область и онтология технологий автоматизации различных видов человеческой деятельности}
\scnidtf{Эволюция технологий проектирования, производства и эксплуатации компьютерных систем и предпосылки создания компьютерных технологий нового поколения}

\scnheader{Предметная область и онтология логико-семантических моделей компьютерных систем, основанных на смысловом представлении информации}
\scnidtf{Предлагаемый подход к построению интеллектуальных компьютерных систем следующего поколения}

\scnheader{Предметная область и онтология внутреннего языка ostis-систем}
\scnidtf{Предметная область и онтология SC-кода (Semantic Computer Code)}
\scnrelfrom{введение}{\textit{\nameref{intro_sc_code}}}

\scnheader{Предметная область и онтология  базовой денотационной семантики SC-кода}
\scniselement{\textit{предметная область и онтология верхнего уровня}}


\scnheader{Предметная область и онтология языка графического представления информационных конструкций, хранимых в памяти ostis-систем}
\scnidtf{Предметная область и онтология SCg-кода (Semantic Code graphical)}
\scnaddhind{1}
\scnrelfrom{введение}{\textit{\nameref{intro_scg}}}
\scnresetlevel

\scnheader{Предметная область и онтология языка линейного представления информационных конструкций, хранимых в памяти ostis-систем}
\scnidtf{Предметная область и онтология SCs-кода (Semantic Code string)}
\scnaddhind{1}
\scnrelfrom{введение}{\textit{\nameref{intro_scs}}}
\scnresetlevel

\scnheader{Предметная область и онтология языка форматирования линейного представления информационных конструкций, хранимых в памяти ostis-систем}
\scnidtf{Предметная область и онтология SCn-кода (Semantic Code natural)}
\scnaddhind{1}
\scnrelfrom{введение}{\textit{\nameref{intro_scn}}}
\scnresetlevel

\scnheader{Предметная область и онтология файлов, внешних информационных конструкций и внешних языков ostis-систем}
\scnrelto{дочерний раздел}{\nameref{intro_lang}}

\scnheader{Предметная область и онтология операционной семантики sc-языка вопросов}
\scnidtf{Предметная область информационно-поисковых действий и агентов, а также соответствующая онтология методов}

\scnheader{Предметная область и онтология операционной семантики логических sc-языков}
\scnidtf{Предметная область и онтология логических исчислений}
\scnidtf{Предметная область и онтология действий и агентов логического вывода, а также соответствующая онтология методов (правил) логического вывода}

\scnheader{Предметная область и онтология sc-языков программирования высокого уровня}
\scnidtf{Предметная область и онтология sc-языков программирования высокого и сверхвысокого уровня, ориентированных на обработку баз знаний ostis-систем}

\scnheader{Предметная область и онтология операционной семантики sc-моделей искусственных нейронных сетей}
\scnidtf{Предметная область и онтология процессов функционирования sc-моделей искусственных нейронных сетей при обработке баз знаний ostis-систем}

\scnheader{Логико-семантическая модель средств автоматизации управления взаимодействием разработчиков различных категорий в процессе проектирования базы знаний ostis-системы}
\scnidtf{Логико-семантическая модель средств автоматизации управления взаимодействием менеджеров, авторов, рецензентов, экспертов и редакторов в процессе проектирования базы знаний ostis-системы}

\scnheader{Предметная область и онтология встроенных ostis-систем поддержки эксплуатации соответствующих ostis-систем конечными пользователями}
\scnidtf{Интеллектуальные \textit{встроенные ostis-системы}, обучающие \textit{конечных пользователей} эффективной эксплуатаии тех \textit{ostis-систем}, в состав которых они входят}
\scnidtf{Предметная область и онтология методов и средств реализации целенаправленного и персонифицированного обучения пользователей каждой ostis-системы}

\scnheader{Предметная область и онтология Экосистемы OSTIS}
\scnidtf{Проект smart-общества}

\scnheader{Логико-семантическая модель Метасистемы IMS.ostis}
\scnrelfrom{примечание}{\scnstartsetlocal

	\bigskip
	\scnfilelong{IMS.ostis}
	\scnrelto{сокращение}{\scnfilelong{Метасистема IMS.ostis}}
	\scnaddlevel{1}
	\scnrelto{сокращение}{\scnfilelong{Intelligent MetaSystem of Open Semantic Technology for Intelligent Systems}}
	\scnaddlevel{-1}
	\scnendstruct}
\scnidtf{Логико-семантическая модель интеллектуального ostis-портала научно-технических знаний по Технологии OSTIS}

\scnendstruct



\end{SCn}

\newpage